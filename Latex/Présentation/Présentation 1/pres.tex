\documentclass[aspectratio=169, 11pt]{beamer}
\usepackage[utf8]{inputenc}
\usepackage[T1]{fontenc}
\usepackage{amsmath}
\usepackage{graphicx}
\usepackage{hyperref}
\usepackage{xcolor}
\usepackage{multicol}

% Set the theme
\usetheme{Warsaw}

% Set the color theme
\usecolortheme{orchid}

% Set the font theme
\usefonttheme{default}

% Define a custom color
\definecolor{myblue}{RGB}{33, 66, 99}
\definecolor{lightblue}{RGB}{80, 133, 244}
\setbeamercolor{structure}{fg=lightblue}

% Use sans serif font as default
\renewcommand{\familydefault}{\sfdefault}

% Change background color
\setbeamercolor{background canvas}{bg=white}

% Title page
\title[Sound Detection and Classification]{\textbf{Sound Detection and Classification}\\ using Spiking Neural Networks}
\author[T. Courrege, L. Gandeel]{Téo Courrege\\Loaï Gandeel}
\date{\today}

% Customize title page
\setbeamertemplate{title page}{
  \begin{centering}
    \vfill
    \textcolor{myblue}{\LARGE\inserttitle}\par
    \vspace{1cm}
    \textcolor{black}{\large\insertauthor}\par
    \vspace{1cm}
    \textcolor{black}{\small\insertdate}\par
    \vfill
  \end{centering}
}

% Customize section page
\AtBeginSection[]{
  \begin{frame}{Outline}
    \begin{multicols}{2}
      \hfill % This will push the table of contents to the right side
      \tableofcontents[currentsection]
    \end{multicols}
  \end{frame}
}

\begin{document}

\begin{frame}[plain]
  \titlepage
\end{frame}

\section{Introduction}

\begin{frame}{Introduction}
  \begin{itemize}
    \item Presentation of the Project: Explaining the motivation and goals of the project.
    \item Sound Detection and Classification: Overview of the importance of sound detection and classification, especially in the context of Spiking Neural Networks (SNNs).
    \item Spiking Neural Networks: Brief introduction to SNNs and their relevance in handling temporal dynamics in sound data.
  \end{itemize}
\end{frame}

\section{Data Used}

\subsection{Choosing the Type of Training Data}

\begin{frame}{Choosing the Type of Training Data}
  \begin{itemize}
    \item Understanding the challenges of training SNNs compared to ANNs.
    \item Criteria for selecting data: Less computationally intensive, adaptable to pre-recorded and real-time processing.
  \end{itemize}
\end{frame}

\subsection{Feasibility}

\begin{frame}{Feasibility}
  \begin{itemize}
    \item Selection of Google AudioSet database.
    \item Addressing copyright concerns (Creative Commons license).
    \item Data collection/extraction: Utilizing GitHub repositories for efficient downloading, formatting, and cropping of sound files.
    \item Assessing resource usage and parallelization impact.
    \item Uncertainties about data quality: Contextual issues, multi-labeling, Weak and Strong Label annotations.
  \end{itemize}
\end{frame}

\subsection{Open to Other Databases}

\begin{frame}{Open to Other Databases}
  \begin{itemize}
    \item Considering alternative databases (Freesound, Kaggle) if the selected dataset is insufficient.
    \item Flexibility in exploring and incorporating additional open-source label databases.
  \end{itemize}
\end{frame}

\section{State of the Art}

\subsection{ANNs Used for Audio Classification}

\begin{frame}{ANNs Used for Audio Classification}
  \begin{itemize}
    \item Overview of existing ANNs for audio classification.
    \item Mentioning pre-trained models on Google AudioSet, rearranged Resnet, inception, densenet, and LSTM-based models.
    \item Providing references to relevant GitHub repositories and research papers.
  \end{itemize}
\end{frame}

\subsection{SNNs Used for Audio Classification}

\begin{frame}{SNNs Used for Audio Classification}
  \begin{itemize}
    \item Overview of existing SNNs for audio classification.
    \item Highlighting spiking convolutional neural networks (SCNN), multi-layer SNN using SpiNNaker, shadow training, and SNN simulators (BindsNET, NEST).
    \item Referencing GitHub repositories and documentation for each model.
  \end{itemize}
\end{frame}

\section{Technical Objectives of the Project}

\subsection{Technical Objectives Achievable with Pre-existing ANNs}

\begin{frame}{Technical Objectives Achievable with Pre-existing ANNs}
  \begin{itemize}
    \item Identifying pre-trained ANNs suitable for achieving good accuracy with Google AudioSet.
  \end{itemize}
\end{frame}

\subsection{Technical Objectives with Data}

\begin{frame}{Technical Objectives with Data}
  \begin{itemize}
    \item Discussing the conversion of analog sound signals to digital representation.
    \item Extracting useful features from audio, including time domain features, frequency-domain features, and spectrograms.
  \end{itemize}
\end{frame}

\subsection{Technical Objectives with SNNs}

\begin{frame}{Technical Objectives with SNNs}
  \begin{itemize}
    \item Finding or creating a pre-trained SNN model for robust sound classification.
    \item Referencing a framework for creating SNNs for sound classification.
  \end{itemize}
\end{frame}

\subsection{Feasibility}

\begin{frame}{Feasibility}
  \begin{itemize}
    \item Addressing uncertainties related to dynamic audio signals, varying acoustic environments, and challenges in modeling temporal aspects.
    \item Discussing potential challenges in training SNNs, considering non-linearity and sparsity of spikes.
    \item Highlighting potential data uncertainties, such as varied audio formats and unavailability of certain samples.
    \item Estimating computation time for data import and training.
  \end{itemize}
\end{frame}

\section{Theoretical Study}

\subsection{Theoretical Study of SNN}

\begin{frame}{Theoretical Study of SNN}
  \begin{itemize}
    \item Exploring different archetypes of spiking neural networks, focusing on the Leaky Integrate and Fire (LIF) model.
    \item Discussing neural coding schemes for converting input pixels into spikes.
    \item Overview of different ways to train SNNs: shadow training, backpropagation using spikes, and local learning rules.
  \end{itemize}
\end{frame}

\section{Pre-processing}

\subsection{Data collection}

\begin{frame}{Pre-processing}
  \begin{itemize}
    \item Collecting the data
    \item Adaptation of the data
    \item Checking that there are no errors / repetitions
  \end{itemize}
\end{frame}

\subsection{Data augmentation}

\section{Spectrograms, MEL and MFCC}
\subsection{Spectrograms}

\begin{frame}{Spectrograms}
  \begin{itemize}
    \item Spectrograms
  \end{itemize}
\end{frame}

\subsection{MEL}

\begin{frame}{MEL}
  \begin{itemize}
    \item MEL
  \end{itemize}
\end{frame}

\subsection{MFCC}

\begin{frame}{MFCC}
  \begin{itemize}
    \item MFCC
  \end{itemize}
\end{frame}

\section{Spiking Neural Networks - First results}

\begin{frame}{Spiking Neural Networks - First results}
  \begin{itemize}
    \item First results
  \end{itemize}
\end{frame}

\section{Conclusion}

\begin{frame}{Conclusion}
  \begin{itemize}
    \item Summary
    \item Future Work
  \end{itemize}
\end{frame}

\end{document}
