\documentclass[11pt]{article}
\usepackage[utf8]{inputenc}
\usepackage{amsmath}
\usepackage{titlesec}
\usepackage{titling}
\usepackage{geometry}
\usepackage{graphicx}
\usepackage{hyperref}
\usepackage{fancyhdr}
\usepackage{wallpaper}
\usepackage{afterpage} 
\usepackage{pagecolor} 
\usepackage{multirow}
\usepackage{wrapfig}
\usepackage{lipsum} % for dummy text
\usepackage{graphicx}
\usepackage[nottoc]{tocbibind} % Put the bibliography in the ToC
\usepackage{url}
\usepackage[toc,page]{appendix}
\usepackage{mdframed}


% Define colors
\usepackage{xcolor}
\definecolor{myblue}{RGB}{33, 66, 99}
\definecolor{mygray}{RGB}{169, 169, 169}
\definecolor{darkbluegrey}{RGB}{44, 62, 80} 

% Page styling
\pagestyle{fancy}
\fancyhf{}
\renewcommand{\headrulewidth}{0pt}
\renewcommand{\footrulewidth}{0pt}
\fancyfoot[C]{\thepage}
\renewcommand{\familydefault}{\sfdefault}

% Define a command for section headers
\titleformat{\section}
  {\color{myblue}\normalfont\Large\bfseries}
  {\color{myblue}\thesection}{1em}{}

% Define a command for subsection headers
\titleformat{\subsection}
  {\color{myblue}\normalfont\large\bfseries}
  {\color{myblue}\thesubsection}{1em}{}

% Adjust page margins
\geometry{a4paper, margin=1in}

% make references clickable
\hypersetup{
    colorlinks=true,
    linkcolor=blue,
    filecolor=magenta,      
    urlcolor=cyan,
}

\begin{document}

% Change the background color of the first page
\pagecolor{darkbluegrey}
\afterpage{\nopagecolor}

% Add a background image
\ThisCenterWallPaper{0.75}{./image/spike_brain.png}

\begin{titlepage}
    \vspace*{\stretch{1}}
    \begin{center}
        \textcolor{white}{\textbf{\Huge Description of Work}}\\ % changed text color to white
        \vspace{1cm}
        \textcolor{white}{\Large Sound Detection and Classification\\using Spiking Neural Networks} % changed text color to white
        \vspace{3cm}
    \end{center}
    \vspace*{\stretch{2}}
    \begin{center}
        \textcolor{white}{ % changed text color to white
            \textbf{COURREGE Téo}\\
            \textbf{GANDEEL Lo'aï}\\
            \vspace{1cm}
            \Large Date: \today}
    \end{center}
    \vspace*{\stretch{1}}
\end{titlepage}

\newpage

\tableofcontents

\pagebreak

\section{Introduction}

Our project addresses the challenge of applying spiking neural networks (SNNs) to audio classification in the field of spiking neural network (SNN) research. This report provides a brief overview of our initial progress in this area.

Inspired by the neural signaling patterns of the human brain, SNNs introduce a temporal element into artificial neural networks. This temporal characteristic positions SNNs as promising candidates for real-time processing and pattern recognition tasks.

Our project specifically addresses audio classification within the broader context of SNNs. Before delving into the details, we outline key aspects including preprocessing, data manipulation/augmentation, initial model implementations, and a look at preliminary results.

\textbf{Preprocessing and Data Manipulation and Augmentation:} Using audio data primarily from the Google AudioSet database, our work involves preprocessing, which includes conversion of signals to image representations, feature extraction, and consideration of encoding schemes suitable for SNNs. Challenges related to data quality, context, and labeling complexity prompted the exploration of data augmentation strategies to improve model robustness.

\section{Data}



\pagebreak

% Bibliography
%\bibliographystyle{siam}
%\bibliography{ref}


\end{document}